\documentclass[11pt,]{article}
\usepackage[margin=1in]{geometry}
\newcommand*{\authorfont}{\fontfamily{phv}\selectfont}
\usepackage[]{mathpazo}
\usepackage{abstract}
\renewcommand{\abstractname}{}    % clear the title
\renewcommand{\absnamepos}{empty} % originally center
\newcommand{\blankline}{\quad\pagebreak[2]}

\providecommand{\tightlist}{%
  \setlength{\itemsep}{0pt}\setlength{\parskip}{0pt}} 
\usepackage{longtable,booktabs}

\usepackage{parskip}
\usepackage{titlesec}
\titlespacing\section{0pt}{12pt plus 4pt minus 2pt}{6pt plus 2pt minus 2pt}
\titlespacing\subsection{0pt}{12pt plus 4pt minus 2pt}{6pt plus 2pt minus 2pt}

\titleformat*{\subsubsection}{\normalsize\itshape}

\usepackage{titling}
\setlength{\droptitle}{-.25cm}

%\setlength{\parindent}{0pt}
%\setlength{\parskip}{6pt plus 2pt minus 1pt}
%\setlength{\emergencystretch}{3em}  % prevent overfull lines 

\usepackage[T1]{fontenc}
\usepackage[utf8]{inputenc}

\usepackage{fancyhdr}
\pagestyle{fancy}
\usepackage{lastpage}
\renewcommand{\headrulewidth}{0.3pt}
\renewcommand{\footrulewidth}{0.0pt} 
\lhead{}
\chead{}
\rhead{\footnotesize MNIST ou le \emph{Hello Word} de la classification d'images -- Projet à rendre le 28/02/2018}
\lfoot{}
\cfoot{\small \thepage/\pageref*{LastPage}}
\rfoot{}

\fancypagestyle{firststyle}
{
\renewcommand{\headrulewidth}{0pt}%
   \fancyhf{}
   \fancyfoot[C]{\small \thepage/\pageref*{LastPage}}
}

%\def\labelitemi{--}
%\usepackage{enumitem}
%\setitemize[0]{leftmargin=25pt}
%\setenumerate[0]{leftmargin=25pt}




\makeatletter
\@ifpackageloaded{hyperref}{}{%
\ifxetex
  \usepackage[setpagesize=false, % page size defined by xetex
              unicode=false, % unicode breaks when used with xetex
              xetex]{hyperref}
\else
  \usepackage[unicode=true]{hyperref}
\fi
}
\@ifpackageloaded{color}{
    \PassOptionsToPackage{usenames,dvipsnames}{color}
}{%
    \usepackage[usenames,dvipsnames]{color}
}
\makeatother
\hypersetup{breaklinks=true,
            bookmarks=true,
            pdfauthor={ ()},
             pdfkeywords = {},  
            pdftitle={MNIST ou le \emph{Hello Word} de la classification d'images},
            colorlinks=true,
            citecolor=blue,
            urlcolor=blue,
            linkcolor=magenta,
            pdfborder={0 0 0}}
\urlstyle{same}  % don't use monospace font for urls


\setcounter{secnumdepth}{0}

\usepackage{color}
\usepackage{fancyvrb}
\newcommand{\VerbBar}{|}
\newcommand{\VERB}{\Verb[commandchars=\\\{\}]}
\DefineVerbatimEnvironment{Highlighting}{Verbatim}{commandchars=\\\{\}}
% Add ',fontsize=\small' for more characters per line
\usepackage{framed}
\definecolor{shadecolor}{RGB}{248,248,248}
\newenvironment{Shaded}{\begin{snugshade}}{\end{snugshade}}
\newcommand{\KeywordTok}[1]{\textcolor[rgb]{0.13,0.29,0.53}{\textbf{#1}}}
\newcommand{\DataTypeTok}[1]{\textcolor[rgb]{0.13,0.29,0.53}{#1}}
\newcommand{\DecValTok}[1]{\textcolor[rgb]{0.00,0.00,0.81}{#1}}
\newcommand{\BaseNTok}[1]{\textcolor[rgb]{0.00,0.00,0.81}{#1}}
\newcommand{\FloatTok}[1]{\textcolor[rgb]{0.00,0.00,0.81}{#1}}
\newcommand{\ConstantTok}[1]{\textcolor[rgb]{0.00,0.00,0.00}{#1}}
\newcommand{\CharTok}[1]{\textcolor[rgb]{0.31,0.60,0.02}{#1}}
\newcommand{\SpecialCharTok}[1]{\textcolor[rgb]{0.00,0.00,0.00}{#1}}
\newcommand{\StringTok}[1]{\textcolor[rgb]{0.31,0.60,0.02}{#1}}
\newcommand{\VerbatimStringTok}[1]{\textcolor[rgb]{0.31,0.60,0.02}{#1}}
\newcommand{\SpecialStringTok}[1]{\textcolor[rgb]{0.31,0.60,0.02}{#1}}
\newcommand{\ImportTok}[1]{#1}
\newcommand{\CommentTok}[1]{\textcolor[rgb]{0.56,0.35,0.01}{\textit{#1}}}
\newcommand{\DocumentationTok}[1]{\textcolor[rgb]{0.56,0.35,0.01}{\textbf{\textit{#1}}}}
\newcommand{\AnnotationTok}[1]{\textcolor[rgb]{0.56,0.35,0.01}{\textbf{\textit{#1}}}}
\newcommand{\CommentVarTok}[1]{\textcolor[rgb]{0.56,0.35,0.01}{\textbf{\textit{#1}}}}
\newcommand{\OtherTok}[1]{\textcolor[rgb]{0.56,0.35,0.01}{#1}}
\newcommand{\FunctionTok}[1]{\textcolor[rgb]{0.00,0.00,0.00}{#1}}
\newcommand{\VariableTok}[1]{\textcolor[rgb]{0.00,0.00,0.00}{#1}}
\newcommand{\ControlFlowTok}[1]{\textcolor[rgb]{0.13,0.29,0.53}{\textbf{#1}}}
\newcommand{\OperatorTok}[1]{\textcolor[rgb]{0.81,0.36,0.00}{\textbf{#1}}}
\newcommand{\BuiltInTok}[1]{#1}
\newcommand{\ExtensionTok}[1]{#1}
\newcommand{\PreprocessorTok}[1]{\textcolor[rgb]{0.56,0.35,0.01}{\textit{#1}}}
\newcommand{\AttributeTok}[1]{\textcolor[rgb]{0.77,0.63,0.00}{#1}}
\newcommand{\RegionMarkerTok}[1]{#1}
\newcommand{\InformationTok}[1]{\textcolor[rgb]{0.56,0.35,0.01}{\textbf{\textit{#1}}}}
\newcommand{\WarningTok}[1]{\textcolor[rgb]{0.56,0.35,0.01}{\textbf{\textit{#1}}}}
\newcommand{\AlertTok}[1]{\textcolor[rgb]{0.94,0.16,0.16}{#1}}
\newcommand{\ErrorTok}[1]{\textcolor[rgb]{0.64,0.00,0.00}{\textbf{#1}}}
\newcommand{\NormalTok}[1]{#1}




\usepackage{setspace}

\title{MNIST ou le \emph{Hello Word} de la classification d'images}
\author{Mohammed Sedki}
\date{Projet à rendre le 28/02/2018}


\begin{document}  

		\maketitle
		
	
		\thispagestyle{firststyle}

%	\thispagestyle{empty}


	\noindent \begin{tabular*}{\textwidth}{ @{\extracolsep{\fill}} lr @{\extracolsep{\fill}}}


\textit{MSP/ES} & \textit{Apprentissage et agrégation de modèles}\\
%Office Hours: TBD  &  Class Hours: TBD\\
%Office: TBD  & Class Room: TBD\\
	&  \\
	\hline
	\end{tabular*}
	
\vspace{2mm}
	


\section{1. Le jeu de données MNIST}\label{le-jeu-de-donnees-mnist}

Le jeu de données MNIST est hébergé sur le page web de Yann LeCun. Ce
jeu de données sert de référence pour les compétitions en apprentissage
où la donnée explicative est sous forme d'image. Commençons par une
visualisation du jeu de données et plus précisémment les \(36\)
premières images. Pour cela nous allons installer le package
\textsf{keras} qui installe à son tour \textsf{tensorflow}. Le block de
code suivant permet cette visualisation

\begin{Shaded}
\begin{Highlighting}[]
\CommentTok{#install.packages("keras", dep=TRUE)}
\KeywordTok{require}\NormalTok{(keras)}
\NormalTok{mnist   <-}\StringTok{ }\KeywordTok{dataset_mnist}\NormalTok{()}
\NormalTok{x_train <-}\StringTok{ }\NormalTok{mnist}\OperatorTok{$}\NormalTok{train}\OperatorTok{$}\NormalTok{x}
\NormalTok{y_train <-}\StringTok{ }\NormalTok{mnist}\OperatorTok{$}\NormalTok{train}\OperatorTok{$}\NormalTok{y}
\NormalTok{x_test  <-}\StringTok{ }\NormalTok{mnist}\OperatorTok{$}\NormalTok{test}\OperatorTok{$}\NormalTok{x}
\NormalTok{y_test  <-}\StringTok{ }\NormalTok{mnist}\OperatorTok{$}\NormalTok{test}\OperatorTok{$}\NormalTok{y}

\CommentTok{# visualize the digits}
\KeywordTok{par}\NormalTok{(}\DataTypeTok{mfcol=}\KeywordTok{c}\NormalTok{(}\DecValTok{6}\NormalTok{,}\DecValTok{6}\NormalTok{))}
\KeywordTok{par}\NormalTok{(}\DataTypeTok{mar=}\KeywordTok{c}\NormalTok{(}\DecValTok{0}\NormalTok{, }\DecValTok{0}\NormalTok{, }\DecValTok{3}\NormalTok{, }\DecValTok{0}\NormalTok{), }\DataTypeTok{xaxs=}\StringTok{'i'}\NormalTok{, }\DataTypeTok{yaxs=}\StringTok{'i'}\NormalTok{)}
\ControlFlowTok{for}\NormalTok{ (idx }\ControlFlowTok{in} \DecValTok{1}\OperatorTok{:}\DecValTok{36}\NormalTok{) \{ }
\NormalTok{  im <-}\StringTok{ }\NormalTok{x_train[idx,,]}
\NormalTok{  im <-}\StringTok{ }\KeywordTok{t}\NormalTok{(}\KeywordTok{apply}\NormalTok{(im, }\DecValTok{2}\NormalTok{, rev)) }
  \KeywordTok{image}\NormalTok{(}\DecValTok{1}\OperatorTok{:}\DecValTok{28}\NormalTok{, }\DecValTok{1}\OperatorTok{:}\DecValTok{28}\NormalTok{, im, }\DataTypeTok{col=}\KeywordTok{gray}\NormalTok{((}\DecValTok{0}\OperatorTok{:}\DecValTok{255}\NormalTok{)}\OperatorTok{/}\DecValTok{255}\NormalTok{), }
        \DataTypeTok{xaxt=}\StringTok{'n'}\NormalTok{, }\DataTypeTok{main=}\KeywordTok{paste}\NormalTok{(y_train[idx]))}
\NormalTok{\}}
\end{Highlighting}
\end{Shaded}

\section{Préparation des données}\label{preparation-des-donnees}

Afin de pouvoir mettre en place les méthodes d'apprentissage classiques,
nous avons besoin d'applatir les images en vecteurs. Cela revient à
transformer une image de dimension \(28 \times 28\) pixels en un vecteur
de dimension \(784\).

\begin{enumerate}
\def\labelenumi{\arabic{enumi}.}
\item
  Préparer le jeu de données en appliquant un applatissement ainsi
  qu'une suppression des pixels nuls pour l'ensemble des images du jeu
  de données. La fonction \textsf{nearZeroVar} du package \textsf{caret}
  permet de repérer les variables de variance nulle.
\item
  Le nombre de variables du jeu de données après applatissement est très
  elevé. Proposer une procédure de réduction de dimension indépendante
  de la méthode d'apprentissage à utiliser. La fonction
  \textsf{preProcess} du package \textsf{caret} permet un ensemble de
  transformation de données.
\end{enumerate}

\begin{Shaded}
\begin{Highlighting}[]
\NormalTok{y_train.freq <-}\StringTok{ }\KeywordTok{table}\NormalTok{(y_train)}
\KeywordTok{barplot}\NormalTok{(y_train.freq)}
\end{Highlighting}
\end{Shaded}




\end{document}

\makeatletter
\def\@maketitle{%
  \newpage
%  \null
%  \vskip 2em%
%  \begin{center}%
  \let \footnote \thanks
    {\fontsize{18}{20}\selectfont\raggedright  \setlength{\parindent}{0pt} \@title \par}%
}
%\fi
\makeatother
