\documentclass[11pt,]{article}
\usepackage[margin=1in]{geometry}
\newcommand*{\authorfont}{\fontfamily{phv}\selectfont}
\usepackage[]{mathpazo}
\usepackage{abstract}
\renewcommand{\abstractname}{}    % clear the title
\renewcommand{\absnamepos}{empty} % originally center
\newcommand{\blankline}{\quad\pagebreak[2]}

\providecommand{\tightlist}{%
  \setlength{\itemsep}{0pt}\setlength{\parskip}{0pt}} 
\usepackage{longtable,booktabs}

\usepackage{parskip}
\usepackage{titlesec}
\titlespacing\section{0pt}{12pt plus 4pt minus 2pt}{6pt plus 2pt minus 2pt}
\titlespacing\subsection{0pt}{12pt plus 4pt minus 2pt}{6pt plus 2pt minus 2pt}

\titleformat*{\subsubsection}{\normalsize\itshape}

\usepackage{titling}
\setlength{\droptitle}{-.25cm}

%\setlength{\parindent}{0pt}
%\setlength{\parskip}{6pt plus 2pt minus 1pt}
%\setlength{\emergencystretch}{3em}  % prevent overfull lines 

\usepackage[T1]{fontenc}
\usepackage[utf8]{inputenc}

\usepackage{fancyhdr}
\pagestyle{fancy}
\usepackage{lastpage}
\renewcommand{\headrulewidth}{0.3pt}
\renewcommand{\footrulewidth}{0.0pt} 
\lhead{}
\chead{}
\rhead{\footnotesize Le jeu de données MINST ou le \emph{Hello word} de la classification
d'images -- Projet à rendre le 28/02/2018}
\lfoot{}
\cfoot{\small \thepage/\pageref*{LastPage}}
\rfoot{}

\fancypagestyle{firststyle}
{
\renewcommand{\headrulewidth}{0pt}%
   \fancyhf{}
   \fancyfoot[C]{\small \thepage/\pageref*{LastPage}}
}

%\def\labelitemi{--}
%\usepackage{enumitem}
%\setitemize[0]{leftmargin=25pt}
%\setenumerate[0]{leftmargin=25pt}




\makeatletter
\@ifpackageloaded{hyperref}{}{%
\ifxetex
  \usepackage[setpagesize=false, % page size defined by xetex
              unicode=false, % unicode breaks when used with xetex
              xetex]{hyperref}
\else
  \usepackage[unicode=true]{hyperref}
\fi
}
\@ifpackageloaded{color}{
    \PassOptionsToPackage{usenames,dvipsnames}{color}
}{%
    \usepackage[usenames,dvipsnames]{color}
}
\makeatother
\hypersetup{breaklinks=true,
            bookmarks=true,
            pdfauthor={ ()},
             pdfkeywords = {},  
            pdftitle={Le jeu de données MINST ou le \emph{Hello word} de la classification
d'images},
            colorlinks=true,
            citecolor=blue,
            urlcolor=blue,
            linkcolor=magenta,
            pdfborder={0 0 0}}
\urlstyle{same}  % don't use monospace font for urls


\setcounter{secnumdepth}{0}

\usepackage{color}
\usepackage{fancyvrb}
\newcommand{\VerbBar}{|}
\newcommand{\VERB}{\Verb[commandchars=\\\{\}]}
\DefineVerbatimEnvironment{Highlighting}{Verbatim}{commandchars=\\\{\}}
% Add ',fontsize=\small' for more characters per line
\usepackage{framed}
\definecolor{shadecolor}{RGB}{248,248,248}
\newenvironment{Shaded}{\begin{snugshade}}{\end{snugshade}}
\newcommand{\KeywordTok}[1]{\textcolor[rgb]{0.13,0.29,0.53}{\textbf{#1}}}
\newcommand{\DataTypeTok}[1]{\textcolor[rgb]{0.13,0.29,0.53}{#1}}
\newcommand{\DecValTok}[1]{\textcolor[rgb]{0.00,0.00,0.81}{#1}}
\newcommand{\BaseNTok}[1]{\textcolor[rgb]{0.00,0.00,0.81}{#1}}
\newcommand{\FloatTok}[1]{\textcolor[rgb]{0.00,0.00,0.81}{#1}}
\newcommand{\ConstantTok}[1]{\textcolor[rgb]{0.00,0.00,0.00}{#1}}
\newcommand{\CharTok}[1]{\textcolor[rgb]{0.31,0.60,0.02}{#1}}
\newcommand{\SpecialCharTok}[1]{\textcolor[rgb]{0.00,0.00,0.00}{#1}}
\newcommand{\StringTok}[1]{\textcolor[rgb]{0.31,0.60,0.02}{#1}}
\newcommand{\VerbatimStringTok}[1]{\textcolor[rgb]{0.31,0.60,0.02}{#1}}
\newcommand{\SpecialStringTok}[1]{\textcolor[rgb]{0.31,0.60,0.02}{#1}}
\newcommand{\ImportTok}[1]{#1}
\newcommand{\CommentTok}[1]{\textcolor[rgb]{0.56,0.35,0.01}{\textit{#1}}}
\newcommand{\DocumentationTok}[1]{\textcolor[rgb]{0.56,0.35,0.01}{\textbf{\textit{#1}}}}
\newcommand{\AnnotationTok}[1]{\textcolor[rgb]{0.56,0.35,0.01}{\textbf{\textit{#1}}}}
\newcommand{\CommentVarTok}[1]{\textcolor[rgb]{0.56,0.35,0.01}{\textbf{\textit{#1}}}}
\newcommand{\OtherTok}[1]{\textcolor[rgb]{0.56,0.35,0.01}{#1}}
\newcommand{\FunctionTok}[1]{\textcolor[rgb]{0.00,0.00,0.00}{#1}}
\newcommand{\VariableTok}[1]{\textcolor[rgb]{0.00,0.00,0.00}{#1}}
\newcommand{\ControlFlowTok}[1]{\textcolor[rgb]{0.13,0.29,0.53}{\textbf{#1}}}
\newcommand{\OperatorTok}[1]{\textcolor[rgb]{0.81,0.36,0.00}{\textbf{#1}}}
\newcommand{\BuiltInTok}[1]{#1}
\newcommand{\ExtensionTok}[1]{#1}
\newcommand{\PreprocessorTok}[1]{\textcolor[rgb]{0.56,0.35,0.01}{\textit{#1}}}
\newcommand{\AttributeTok}[1]{\textcolor[rgb]{0.77,0.63,0.00}{#1}}
\newcommand{\RegionMarkerTok}[1]{#1}
\newcommand{\InformationTok}[1]{\textcolor[rgb]{0.56,0.35,0.01}{\textbf{\textit{#1}}}}
\newcommand{\WarningTok}[1]{\textcolor[rgb]{0.56,0.35,0.01}{\textbf{\textit{#1}}}}
\newcommand{\AlertTok}[1]{\textcolor[rgb]{0.94,0.16,0.16}{#1}}
\newcommand{\ErrorTok}[1]{\textcolor[rgb]{0.64,0.00,0.00}{\textbf{#1}}}
\newcommand{\NormalTok}[1]{#1}




\usepackage{setspace}

\title{Le jeu de données MINST ou le \emph{Hello word} de la classification
d'images}
\author{Mohammed Sedki}
\date{Projet à rendre le 28/02/2018}


\begin{document}  

		\maketitle
		
	
		\thispagestyle{firststyle}

%	\thispagestyle{empty}


	\noindent \begin{tabular*}{\textwidth}{ @{\extracolsep{\fill}} lr @{\extracolsep{\fill}}}


\textit{MSP/ES} & \textit{Apprentissage et agrégation de modèles}\\
%Office Hours: TBD  &  Class Hours: TBD\\
%Office: TBD  & Class Room: TBD\\
	&  \\
	\hline
	\end{tabular*}
	
\vspace{2mm}
	


\section{1. Jeu de données}\label{jeu-de-donnees}

L'étude de la diversité génétique humaine présente un intérêt pour
divers domaines allant de la compréhension génétique des maladies aux
applications en criminologie. La détection de sous-populations ou
clusters est la clé pour la reconstruction de l'histoire démographique
d'une population. On s'intéresse au regroupement en sous-populations
(clustering) du jeu de données \emph{Human Genome Diversity Panel}
disponible sur le site \url{http://www.cephb.fr/hgdp/}. Ce jeu de
données est composé de \(1043\) individus et \(660918\) marqueurs SNP
(bialléliques). L'accès à ce jeu de données se fait via l'installation
du package \textsf{HGPD.CEPH} comme suit

\begin{Shaded}
\begin{Highlighting}[]
\KeywordTok{install.packages}\NormalTok{(}\StringTok{"HGDP.CEPH"}\NormalTok{, }\DataTypeTok{repos=}\StringTok{"https://genostats.github.io/R/"}\NormalTok{)}
\KeywordTok{require}\NormalTok{(HGDP.CEPH)}
\CommentTok{# lire données}
\NormalTok{filepath <-}\KeywordTok{system.file}\NormalTok{(}\StringTok{"extdata"}\NormalTok{, }\StringTok{"hgdp_ceph.bed"}\NormalTok{, }\DataTypeTok{package=}\StringTok{"HGDP.CEPH"}\NormalTok{)}
\NormalTok{x <-}\StringTok{ }\KeywordTok{read.bed.matrix}\NormalTok{(filepath)}
\CommentTok{# données SNP et individus}
\KeywordTok{head}\NormalTok{(x}\OperatorTok{@}\NormalTok{snps)}
\KeywordTok{head}\NormalTok{(x}\OperatorTok{@}\NormalTok{ped)}
\end{Highlighting}
\end{Shaded}

\section{2. Analyse en composantes
principales}\label{analyse-en-composantes-principales}

L'analyse en composantes principales est communément utilisée pour
visualiser des groupes dans un nuage de points. Avant d'appliquer une
telle procédure, nous avons besoin de normaliser les données. Rappelons
que les données sont sous forme d'une matrice de tailles \((n,p)\) où
\(n\) est le nombre d'individus observés et \(p\) le nombre de marqueurs
observés pour chaque individu. Chaque marqueur est un SNP, qui possède
deux allèles possibles. Ainsi, le génotype d'un marqueur particulier
peut être codé sur la base du nombre d'allèles (0, 1 ou 2).

\begin{enumerate}
\def\labelenumi{\alph{enumi}.}
\item
  Rappeler l'espérance et la variance d'un marqueur \(X_j\) observé sous
  l'équilibre de Hardy-Weinberg. Proposer deux estimateurs intuitifs de
  ces deux quantités.
\item
  Le package \textsf{gaston} propose une fonction qui automatise le
  calcul des estimateurs de ces deux quantités sur la matrice de
  données. Vérifier que les estimateurs calculés par cette fonction sur
  la première colonne de la matrice de données correspondent aux
  estimateurs proposés en réponse à la question précédente.
\end{enumerate}

On note \(\mathbf X\) la matrice de données obtenue après
standardisation des colonnes du jeu de données. La décomposition en
valeurs singulière de \(\mathbf X\) donnée par

\[\mathbf X = U \Gamma V^t \,\] où \(\Gamma\) est une matrice diagonale
formée par les valeurs dites \emph{sigulières}
\(\gamma_1, \gamma_2,\ldots\).

On note \(S\), la matrice de covariance empirique des marqueurs de
taille \(p \times p\) définie par

\[S = \frac{1}{n-1} \mathbf X^t \mathbf X\]

Les vecteurs propres \(\mathbf v_1, \mathbf v_2, \ldots\) associés aux
valeurs propres \(\lambda_1 \ge \lambda_2 \ge \ldots\) où
\(\lambda_i = \gamma_i^2\) sont les composantes principales\footnote{La
  matrice \(S\) correspond à la matrice de corrélation lorsque les
  colonnes de \(\mathbf X\) sont centrées et réduites (standardisées).}.
En génétique des populations où \(n < p\), {[}@Cavalli{]} se sont
intéressés à la matrice dite \emph{duale} de taille \(n\times n\) donnée
par \[H = \frac{1}{p} \mathbf X \mathbf X^t,\] pour mettre en place une
analyse en composantes principales. On notera \(\xi_i\) les valeurs
propres associées à \(H\) et \(u_i\) ses vecteurs propres.

\begin{enumerate}
\def\labelenumi{\alph{enumi}.}
\setcounter{enumi}{2}
\item
  Rappeler le lien entre les éléments propres des matrices \(S\) et
  \(H\).
\item
  La fonction \textsf{GRM} du package \textsf{gaston} permet le calcul
  de la matrice \(H\). À l'aide de la fonction \textsf{select.snps},
  restreindre l'étude aux snp autosomaux avec une fréquence de l'allèle
  mineur strictement supérieure à \(0.05\).
\item
  Calculer les vecteurs propres associés à la matrice \(H\) à l'aide de
  la fonction \textsf{eigen}. Représenter les individus du jeu de
  données dans le premier plan factoriel. Colorier chaque point en
  fonction de sa région indiqué dans \textsf{x@ped@region7}.
\item
  Regrouper le jeu de données en \(7\) groupes à l'aide d'un
  \textsf{kmeans} sur les deux premiers vecteurs propres. Comparer la
  partition obtenue à la partition \textsf{x@ped@region7} à l'aide d'une
  matrice de confusion. Évaluer la correspondance entre les deux
  partitions avec la fonction \textsf{ARI} du package
  \textsf{VarSelLCM}.
\end{enumerate}

\section{3. Clustering spectral}\label{clustering-spectral}

Pour introduire le sujet nous avons besoin du vocabulaire de la théorie
des graphes. Pour \(n\) points, on définit un graphe \(G\) de sommets
\(\{1,\ldots,n\}\) On associe à \(G\) une matrice \(W\) qui décrit la
force de lien entre ses sommets : les couples de sommets \((i,j)\)
correspondant à une grande valeur \(w_{i,j}\) sont fortement connectés.
Les entrées nulles \(w_{i,j} = 0\) indiquent les sommets non-connectés
de \(G\). Ce type de modélisation est d'une grande flexibilité par la
variété de matrices de similarité qu'on peut définir sur un jeu de
données à condition que celle-ci soit symétrique et à valeurs positives.
L'idée du clustering spectral est de détecter les composantes connexes
du graphe à l'aide du spectre d'un Laplacien discret. Il existe
différentes manières de calculer un Laplacien discret sur un graphe.
Nous allons faire appel à la version dite \emph{normalisée} donnée par

\[\mathcal L =  D^{-\frac{1}{2}} L D^{-\frac{1}{2}},\] où \(L = D-W\) et
\(D= diag(d_1, \ldots, d_n)\) est la matrice diagonale où
\(d_i = \sum_{j=1}^n w_{ij}\) correspond au degré du sommet \(i\).

L'article de {[}@lee2010{]} propose d'adopter l'approche du clustering
spectral dans le contexte de génétique des populations. Cette approche
permet une détection automatique du nombre de sous-populations ainsi que
le clustering des individus.

\begin{enumerate}
\def\labelenumi{\alph{enumi}.}
\setcounter{enumi}{6}
\item
  Résumer l'idée à l'origine de l'heuristique du choix du nombre de
  vecteurs propres du Laplacien ainsi que le nombre de sous-populations.
\item
  Reprendre le jeu de données restreint aux snp autosomaux avec une
  fréquence de l'allèle mineur strictement supérieure à \(0.05\).
  Implémenter la procédure décrite dans {[}@lee2010, Algorithm 1 (de 1:
  à 7:){]}\footnote{Attention: la définition du Laplacien normalisé à
    l'étape (3:) est erronée . Utiliser la définition donnée dans cet
    énoncé.} en appliquant un algorithme des \textsf{kmeans} sur la
  représentation des données obtenues. Expliciter les étapes ainsi que
  le résultat obtenu. Comparer la partition obtenue à la partition
  \textsf{x@ped@region7} à l'aide d'un matrice de confusion. Évaluer la
  correspondance entre les deux partitions avec la fonction \textsf{ARI}
  du package \textsf{VarSelLCM}. Conclure.
\end{enumerate}




\end{document}

\makeatletter
\def\@maketitle{%
  \newpage
%  \null
%  \vskip 2em%
%  \begin{center}%
  \let \footnote \thanks
    {\fontsize{18}{20}\selectfont\raggedright  \setlength{\parindent}{0pt} \@title \par}%
}
%\fi
\makeatother
